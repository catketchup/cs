% Created 2020-02-29 Sat 18:19
% Intended LaTeX compiler: pdflatex
\documentclass[11pt]{article}
\usepackage[utf8]{inputenc}
\usepackage[T1]{fontenc}
\usepackage{graphicx}
\usepackage{grffile}
\usepackage{longtable}
\usepackage{wrapfig}
\usepackage{rotating}
\usepackage[normalem]{ulem}
\usepackage{amsmath}
\usepackage{textcomp}
\usepackage{amssymb}
\usepackage{capt-of}
\usepackage{hyperref}
\usepackage{minted}
\author{user name}
\date{\today}
\title{}
\hypersetup{
 pdfauthor={user name},
 pdftitle={},
 pdfkeywords={},
 pdfsubject={},
 pdfcreator={Emacs 26.2 (Org mode 9.1.9)}, 
 pdflang={English}}
\begin{document}

\tableofcontents

Emacs note
\section{problems}
\label{sec:orge5e3682}
\begin{itemize}
\item emacs/ein markdown latex equations
\item transplant emacs;//
\item yilun's extension;
\item open website in emacs, eww/Xwidget Webkit
\item install an exention without using emacs
\item sougou input and emacs
\item centaur emacs github
\end{itemize}
\section{log}
\label{sec:orga9c08ec}
\subsection{08/2019}
\label{sec:org95ed809}
08/
\begin{itemize}
\item c-c j: go to function in the repository
\item c-c f: go to function in the file
\item c-x n \ldots{}: narrow
\item c-j: enter
\item c-': multicusor
\item c-;: backspace
\item electric pair mode
\item show paren mode
\item smartparen mode
\item doom-theses
\item doom-modeline
\item defun move-bottom-subtree: c-x c-j
\item emacs/scratch: c-c s
\item ein mode
\item ein markdown
\end{itemize}

08/28
\begin{itemize}
\item doom-nova theme
\item beacon mode
\item magic-latex-buffer
\end{itemize}

\subsection{09/2019}
\label{sec:org22bdbe7}
09/08,09,10,11,12,13,15
\begin{itemize}
\item ?: latex display buffer; cmu guy,
\item ?: German input? extension package load;
\item org source code;

\item emacs configuration, munen; from helm to ivy, swiper, abo-abo;

\item ?: eshell shortkey?
\item describe-function; run shell scripts in emacs(query?);

\item ?: shell-script mode pass parameter
\end{itemize}

09/15
\begin{itemize}
\item configured emacs on the old laptop;
\item ?: Centaur emacs; require and install;
\end{itemize}

09/16,17,18,20,22,23,26
\begin{itemize}
\item Centaur emacs; rgrep; provide;

\item people;

\item python-mode; dired; read-only;

\item centaur, anaconda mode; problems;

\item sogou input in emacs; backspace, M-,;cl

\item irc;
\item ?:ein-mode plot pop up;
\item ?:lpy-mode; .el file read only; errors; github;  configuration; kill buffers;
\item ?:startup page fo centaur; auto log in circe; summary on how to debug in emacs;
\end{itemize}
\subsection{02/2020}
\label{sec:org3b6aa65}
02/15
\begin{itemize}
\item PDFView
\end{itemize}

02/29
\begin{itemize}
\item changed centaur-theme to be "default" in custom.el
\item commented out "circe"
\end{itemize}


\section{topics1}
\label{sec:orga1d6d65}
\subsection{keys}
\label{sec:org7c9b861}
\begin{itemize}
\item \url{https://www.emacswiki.org/emacs/EmacsNewbieKeyReference}
\end{itemize}
\subsection{move cursor}
\label{sec:orgfe28edd}
\begin{itemize}
\item c-f: forward one character
\item c-b: backward one character
\item c-n: down one screen line
\item c-p: up one screen line
\item c-a: the begining of the line
\item c-e: the end of the line

\item m-f: forward one word
\item m-b: backward one word
\item c-c c-p: move up to the closest entry
\item c-c c-n: move down to the closest entry
\item c-c c-u: move up to the parent heading
\item c-c c-f: move forward to the next heading at the same level

\item c-v: page down
\item m-v: page up
\item m-s->: to the top of the file
\item m-s-<: to the bottom of the file

\item m-x: emacs command
\item c-o: switch in the emacs command region
\end{itemize}
\subsection{text}
\label{sec:org550fcce}
\begin{itemize}
\item c-w: cut the text
\item m-w: copy the text
\item c-k: kill a line
\item c-y: paste the text
\item c-d: delete
\item c-/: undo
\item c-s: search
\item c-e: export
\end{itemize}
\subsection{buffer}
\label{sec:orga44919a}
\begin{itemize}
\item c-x b: go to a buffer
\item c-x k: kill the buffer
\item c-x d: kill the buffer in the command place
\end{itemize}
\subsection{file}
\label{sec:org59d0d96}
\begin{itemize}
\item c-x c-f: find files
\item c-x c-s: save the file
\item c-x c-b: buffer list
\item c-x b: move to another buffer
\item c-x c: exit
\end{itemize}
\subsection{frame}
\label{sec:org697aad3}
\begin{itemize}
\item c-x 3: left/right split
\item c-x 2: up/down split
\item c-x 0: close the current frame
\item c-x 1: maximize the current frame
\item c-x o: go to the other frame
\end{itemize}
\subsection{link}
\label{sec:org30bdcf6}
\begin{itemize}
\item c-c l: capture link
\item c-c c-l: insert link
\item c-c c-o: open the link
\end{itemize}
\subsection{functions}
\label{sec:org8f81705}
\begin{itemize}
\item use scratch to run code temperal
\item c-h k: show the function of a key(m-x describe-key)
\item c-h f: (m-x describe-function)
\item comment or uncomment
\item comment box
\item read pdf and make note:
\item IRC
\end{itemize}
\subsection{lisp}
\label{sec:org164dbee}
\subsection{eshell}
\label{sec:orga5f4467}
\begin{itemize}
\item \url{https://www.masteringemacs.org/article/complete-guide-mastering-eshell}
\end{itemize}
\subsection{help}
\label{sec:orgeb827f9}
\begin{itemize}
\item 
\end{itemize}
\subsection{extensions and modes}
\label{sec:org2bdbc28}
\begin{itemize}
\item find the source codes in \textasciitilde{}/.emacs.d/elpa/
\end{itemize}
\subsubsection{installation list}
\label{sec:org0d464b6}
\begin{itemize}
\item helm
\item company
\item magit
\item electric pair mode
\item show paren mode
\item smartparen mode
\item doom-theses
\item doom-modelines
\item scratch
\item magic-latex-buffer
\item beacon
\item ivy/counsel/swiper
\end{itemize}
\subsubsection{melpa}
\label{sec:orgb63df30}
\begin{itemize}
\item \url{https://blog.csdn.net/sjhuangx/article/details/51252522}
\item update: M-x package-refresh-contents
\end{itemize}
\subsubsection{configuration:}
\label{sec:org3c8a3b5}
\begin{itemize}
\item \url{https://www.cnblogs.com/morole/p/9965685.html}
\item \url{https://github.com/munen/emacs.d} (very good configuration example)
\item installed parts of extension recommanded by him
\end{itemize}
\subsubsection{hook}
\label{sec:orgf141269}
\subsubsection{helm}
\label{sec:org8492d50}
\subsubsection{Magit}
\label{sec:org4f2da51}
\begin{itemize}
\item c-x g: open Magit
\item s: stage files
\item c c: commit and make comment
\item c-c c-c: complete commit
\item P u: push to the remote orgin
\end{itemize}
\subsubsection{company}
\label{sec:org7fa2e00}
\subsubsection{yas-snippet}
\label{sec:orgbeef690}
\subsubsection{py-autopep8: \url{https://github.com/paetzke/py-autopep8.el}}
\label{sec:org58e0562}
\subsubsection{linum(show line numbers):}
\label{sec:org751305c}
\begin{itemize}
\item \url{http://ergoemacs.org/emacs/emacs\_line\_number\_mode.html}
\end{itemize}
\subsubsection{ein: run jupyter notebook in emacs}
\label{sec:org942c92c}
\begin{itemize}
\item ein:jupyter-server-start
\item ein:stop
\item C-u-c-b/a: add markdown
\end{itemize}
\begin{enumerate}
\item auto-complete
\label{sec:orgfc1fe34}
o(add-hook 'ein:notebook-mode-hook \#'anaconda-mode)

(defun user-ein-reply-callback (args content -metadata-not-used-)
  (let ((callback (plist-get args :callback))
        (candidates (plist-get content :matches)))
    (funcall callback candidates)))

(defun user-company-ein-callback (callback)
  (ein:kernel-complete
   (ein:get-kernel)
   (thing-at-point 'line)
   (current-column)
   (list :complete\(_{\text{reply}}\)
         (cons \#'user-ein-reply-callback (list :callback callback))))
  )

(defun user-company-ein-backend (command \&optional arg \&rest ignored)
  (interactive (list 'interactive))
  (case command
    (interactive (company-begin-backend 'user-company-ein-backend))
    (prefix (company-anaconda-prefix))
    (candidates (cons :async \#'user-company-ein-callback))
    (location nil)
    (sorted t)
    )
  )
\end{enumerate}

\subsubsection{evil}
\label{sec:org21b883b}
\subsubsection{elpy}
\label{sec:org432f3a5}
\subsubsection{flycheck}
\label{sec:org2009ceb}
\subsubsection{sphinx-doc: \url{https://github.com/naiquevin/sphinx-doc.el}}
\label{sec:org8e6ca7b}
\subsubsection{counsel}
\label{sec:org01c34ff}
\subsubsection{browse and tag code: \url{https://zhuanlan.zhihu.com/p/67312736}}
\label{sec:org381f439}

\subsubsection{ivy}
\label{sec:org377499e}
\subsubsection{GNU global}
\label{sec:org8117b66}
\subsubsection{ggtags: \url{https://github.com/leoliu/ggtags}}
\label{sec:org053109d}
\subsubsection{ctags?}
\label{sec:orgaa67e35}
\subsubsection{grep?}
\label{sec:orgec23ac5}
\subsubsection{Speedbar: browse source tree}
\label{sec:org5c96d04}
\begin{itemize}
\item SPC: open the children of a node
\item RET: open the node
\item U: go up parent directory
\item n or p: move to next or previous node
\item m-p or m-p: move to next or previous node at the current level
\item b: switch to buffer list using Speedbar presentation
\item f: switch back to file list
\end{itemize}
\subsubsection{sr-speedbar}
\label{sec:org2658046}
\subsubsection{pyim: \url{https://github.com/tumashu/pyim}}
\label{sec:org3f7a57f}
\subsubsection{projectile}
\label{sec:org0278126}
\begin{itemize}
\item \url{https://github.com/bbatsov/projectile}
\item \url{https://projectile.readthedocs.io/en/latest/usage/}: usage
\end{itemize}
\subsubsection{org mode}
\label{sec:org45d6736}
\begin{itemize}
\item c-c c-x c-l: show equation
\item c-c c-c: hide equation
\item c-c c-e l l: Export as \LaTeX{} file myfile.tex
\item C-c C-e l p: Export as \LaTeX{} and then process to PDF
\item C-c C-e l o: Export as \LaTeX{} and then process to PDF, then open the resulting PDF file
\end{itemize}
\subsubsection{python mode}
\label{sec:org771dcc3}
\begin{itemize}
\item \url{https://realpython.com/emacs-the-best-python-editor/}
\item c-c c-p: open process windowm
\item c-c c-c: run python file
\item python mode command: \url{https://stackoverflow.com/questions/25669809/how-do-you-run-python-code-using-emacs}
\item C-c-j jump to def/class
\item C-c-f find file
\end{itemize}
\subsubsection{c/c++ mode}
\label{sec:orge4f6cb4}
\begin{itemize}
\item \url{http://top.jobbole.com/}
\end{itemize}
\subsubsection{shell-script mode}
\label{sec:org2102631}
\begin{itemize}
\item c-c c-x execute script
\end{itemize}
\subsubsection{javascript mode}
\label{sec:org207f1fc}
\subsubsection{tex mode}
\label{sec:org5507c6e}
\subsubsection{gnu global and emacs}
\label{sec:org1b0f87c}
\begin{itemize}
\item \url{https://www.cnblogs.com/elvalad/p/4069656.html}
\item \url{https://www.gnu.org/software/global/download.html}
\item \url{https://www.cnblogs.com/elvalad/p/4069656.html}
\item \url{https://blog.csdn.net/gatieme/article/details/78819740}
\end{itemize}
\subsubsection{browser}
\label{sec:org01ffad8}
\begin{itemize}
\item \url{https://www.emacswiki.org/emacs/BrowseUrl}
\end{itemize}
\subsubsection{miscellaneous}
\label{sec:org6bdc3e7}
\subsubsection{theme}
\label{sec:orgf071d88}
\subsubsection{latex}
\label{sec:org1c2ead3}
1.C-c C-c and choose latex compiler and compile
2.C-c C-c and choose bibtex to compile the .bib file
3.compile for another 2 time
\section{topics2}
\label{sec:orgf6320a9}
\subsection{dired(file manager)}
\label{sec:orga80bc2a}
\subsection{eshell}
\label{sec:org3b743d7}
\subsection{upgrade emacs}
\label{sec:org319776d}
\begin{itemize}
\item sudo add-apt-repository ppa:kelleyk/emacs
\item sudo apt upgrade
\item sudo apt install emacs26
\item sudo apt remove --autoremove emacs26 emacs26-nox
\end{itemize}
\subsection{from helm to ivy}
\label{sec:org4b20588}
\begin{itemize}
\item \url{https://sam217pa.github.io/2016/09/13/from-helm-to-ivy/\#fn:2}
\end{itemize}
\subsection{read-only-mode}
\label{sec:orgb306f1d}
\begin{itemize}
\item read-only-mode
\item revert-buffer
\item 
\end{itemize}
\subsection{PDFView}
\label{sec:org6d85955}
\begin{itemize}
\item C-c C-c: view pdf
\end{itemize}
\section{people}
\label{sec:org47eb61c}
\subsection{Richard Stallman}
\label{sec:orgf74b140}
\begin{itemize}
\item founder of the GNU Project and author of GNU Emacs
\end{itemize}
\subsection{manateelazycat(王勇)}
\label{sec:org8c24e8c}
\begin{itemize}
\item \url{https://manateelazycat.github.io/index.html}
\item 
\end{itemize}
\subsection{important hakers}
\label{sec:orgeb4a7d5}
\begin{itemize}
\item \url{https://manateelazycat.github.io/emacs/2019/05/12/emacs-hackers.html}
\end{itemize}
\section{centaur emacs}
\label{sec:org9a4b5a0}
\begin{itemize}
\item \url{https://github.com/seagle0128/.emacs.d} (centaur emacs, Vincent Zhang)
\item init-swiper
\item wrap
\item backspace C-;
\item backup-directory-alist
\item turn on/off flyspell/flycheck
\item ansi-term: line mode and character mode
\item magic-latex
\item German input
\item youdao
\item show-paren-mode
\item emacs client
\end{itemize}

python-mode
\begin{itemize}
\item auto-pep8//yapf
\item anaconda, python version
\item setq and config

\item ?: babel
\item ?: config read-only mode and hooks
\item ?: remind the routines in the modules
\item ?: swiper cursor

\item ?: sphinx-doc
\item ?: multicusor
\item ?: auto-complete
\item ?: projectile
\item ?: python functionality
\item ?: ein mode: can't open ipynb
\item ?: anaconda-mode
\item 


\item ?: sogou Chinese input
\item ?: org-mode latex preview
\item ?: font size
\end{itemize}

\section{how to debug in emacs?}
\label{sec:org113c09a}
\begin{itemize}
\item scratch
\item message
\item rgrep key words
\item find the function definations
\end{itemize}
[must read the docs and comments carefully]
\begin{itemize}
\item ask yilun
\item ask the author

\item comment certain packages and see the difference
\end{itemize}
\end{document}